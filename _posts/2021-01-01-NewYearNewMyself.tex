---
layout:     post
title:      写在2021之前
subtitle:   Do yourself!
date:       2021-01-01
author:     Henry
header-img: img/post-bg-miui6.jpg
catalog: true
tags:
    - Myself
---


## 前言
Hi,很久没有去写过blog,在过去的2020年个人觉得是很有收获的一年,至少在思想上获得理性,能够更加理智的看待问题,而对于自己的思想也有了更加深刻的认识。

在过去的2020年中,我个人感觉到的是一种在很忙碌之后所能体会到的一种快乐,在《少有人走过的路》中所讲到的一种推迟的满足感去获得一种长久的快乐的感觉,在这一年开始能够找到一种对于事物的把控感,能够脱离自己物理限制,从另一个更高维度去看待自己所面临的处境从而获得一种更加坚定的力量是我对这一年中收获很大的一个东西。而除了在这一年中对于无论是个人能力或是较为片面的物质上,都在这一年中获得了自己比较满意的结果。

在另一个方面这一年个人也是收获到了对于一种自我的思考,更加懂得了一种为人的局限性,在以前一直感觉人有着无限的可能,只要我想do it and then everything can be done,像是我的性格,我的喜好与个人对于事物与时间的掌控能力,我在这一年中开始接触了很多不一样的声音,我认为对我影响比较大例如ksam man让我更加理性思考,tengo让我对于这一个世界有着更加不一样的认识,像是podcast的遥遥无期让我再次找到自己想要成为一个什么样的人,或是不在场让我个人兴趣得以满足等,他们确实给了我很大的力量,我也感觉到这一世间或是正是由很多的不确定互相的交织于此,才塑造了一个非常独特的自我,在这一年或是也是找到了新的自己,开始能够认识到自己是一个什么样的人,学会懂得自己自我性格的独特性,去放下那一个一直希望去做一个合群的自己,我开始明白或许性格都是个人成长与基因所决定,我所能做的是利用好自己的内心的力量去勇敢正视自己的内心优点与不足,利用好自我内心的力量。

而对于2021,我更是希望自己能够去做好一个敢于并能够投身到竞争中的人,我想我对于自己思想与性格有着更加深入的理解,这种特殊的力量开始让我能够放下对于自己过去的执拗而选择去正视自我,我更加应该改变过去对于自我理性充满浪漫主义色彩的偏执,更加应该去变得更加实际更具有适应性去更快的获得一种在社会所定义的成功,不能再被自己的理想主义所占据自我而最终使得自己总是获得一种怀才不遇的感受,更应该的是在获得自己想要的之后回归自我的内心所在。

2021年我想:
+ 多阅读,听有价值的节目,并多做笔记与回顾,每周保持博客思想类的周更;
+ 技术总结,在技术能力上多积累,利用好自己的工具,像周报一样保持最少一周一更,最好是可以在每次学习完后坚持写好思维导图或是MD/LaTex笔记,只有多多总结回顾知识才能是自己的;
+ 减少使用手机次数,自己要明白手机只是娱乐时代精神控制的途径,不要被同化,需要保持自己对于这一个世界的独特思考和对于各种事件自己的观点与认识;
+ 在遇到困难时能够保持自己理性、理性、理性,尽可能找出问题的关键,不要回避去做出最优的选择;
+ 一些指标类的数字:1)4.0(大三); 2)2W+10\%; 3)一天需要保持7小时的睡眠时间; 4)每周3次的1.2km跑步;5)20本书加上笔记导图;5)获得一级竞赛的一等奖及其以上;6)更加努力积极的参与自己想要的竞争。
